% Document Definition
\documentclass[11pt, twocolumn]{article}
\usepackage[letterpaper, left=1in, right=1in, top=1in, bottom=1in]{geometry}
\usepackage{amscd,amssymb,amsmath,amsthm,mathtools}
\usepackage{kpfonts}
\usepackage{titling}

\renewcommand{\baselinestretch}{1.5}
\setlength{\parindent}{0pt}
\pagenumbering{gobble}

\setlength{\droptitle}{-6em}
\title{\Large\textbf{Determining a Cache Hit/Miss with RDMA -- A NetCAT Replication} \\
CS6465 -- Final Project}
\author{Emerson Ford\hspace{20pt}Calvin Lee}
\posttitle{\par\end{center}}
\predate{}
\postdate{}
\date{}

\begin{document}
\maketitle
\section{Overview}

\section{Investigations}
We plan to utilize either CloudLab or spare hardware available at the Center for High Performance Computing to replicate NetCAT.
We would attempt to recreate the same environment described in the paper using Infiniband, DDIO, RDMA, etc first on bare-metal hardware.
Then, we would work to show that we can pull basic information about DDIO-mapped cache state (ignoring fancier attacks like keystroke predictions).
From there, we would migrate this work over to VMs and run similar experiments.

\section{Results}

\section{Problems}



\end{document}
